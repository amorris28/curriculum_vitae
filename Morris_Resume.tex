%% start of file `template.tex'.
%% Copyright 2006-2015 Xavier Danaux (xdanaux@gmail.com).
%
% This work may be distributed and/or modified under the
% conditions of the LaTeX Project Public License version 1.3c,
% available at http://www.latex-project.org/lppl/.


\documentclass[11pt,letterpaper,sans]{moderncv}        % possible options include font size ('10pt', '11pt' and '12pt'), paper size ('a4paper', 'letterpaper', 'a5paper', 'legalpaper', 'executivepaper' and 'landscape') and font family ('sans' and 'roman')

% moderncv themes
\moderncvstyle{classic}                             % style options are 'casual' (default), 'classic', 'banking', 'oldstyle' and 'fancy'
\moderncvcolor{blue}                               % color options 'black', 'blue' (default), 'burgundy', 'green', 'grey', 'orange', 'purple' and 'red'
%\renewcommand{\familydefault}{\sfdefault}         % to set the default font; use '\sfdefault' for the default sans serif font, '\rmdefault' for the default roman one, or any tex font name
%\nopagenumbers{}                                  % uncomment to suppress automatic page numbering for CVs longer than one page

% character encoding
\usepackage[utf8]{inputenc}                       % if you are not using xelatex ou lualatex, replace by the encoding you are using
%\usepackage{CJKutf8}                              % if you need to use CJK to typeset your resume in Chinese, Japanese or Korean

% adjust the page margins
\usepackage[scale=0.8]{geometry}
%\setlength{\hintscolumnwidth}{3cm}                % if you want to change the width of the column with the dates
%\setlength{\makecvtitlenamewidth}{10cm}           % for the 'classic' style, if you want to force the width allocated to your name and avoid line breaks. be careful though, the length is normally calculated to avoid any overlap with your personal info; use this at your own typographical risks...

% personal data
\name{Andrew}{Morris}
\title{Postdoctoral Scholar}                               % optional, remove / comment the line if not wanted
%\address{1573 Wilson St.}{Eugene, OR 97402}{USA}% optional, remove / comment the line if not wanted; the "postcode city" and "country" arguments can be omitted or provided empty
\phone[mobile]{+1~(860)~670~4130}                   % optional, remove / comment the line if not wanted; the optional "type" of the phone can be "mobile" (default), "fixed" or "fax"
%\phone[fixed]{+2~(345)~678~901}
%\phone[fax]{+3~(456)~789~012}
\email{amorris3@uoregon.edu}                               % optional, remove / comment the line if not wanted
\homepage{ahmorris.org}                         % optional, remove / comment the line if not wanted
\social[linkedin]{andrew-morris-71033a41}                        % optional, remove / comment the line if not wanted
\social[twitter]{ahmorris1}                             % optional, remove / comment the line if not wanted
\social[github]{amorris28}                              % optional, remove / comment the line if not wanted
%\extrainfo{additional information}                 % optional, remove / comment the line if not wanted
%\photo[64pt][0.4pt]{avatar}                       % optional, remove / comment the line if not wanted; '64pt' is the height the picture must be resized to, 0.4pt is the thickness of the frame around it (put it to 0pt for no frame) and 'picture' is the name of the picture file
%\quote{Some quote}                                 % optional, remove / comment the line if not wanted

% bibliography adjustements (only useful if you make citations in your resume, or print a list of publications using BibTeX)
%   to show numerical labels in the bibliography (default is to show no labels)
\makeatletter\renewcommand*{\bibliographyitemlabel}{\@biblabel{\arabic{enumiv}}}\makeatother
\makeatletter
\renewcommand*{\makeletterclosing}{
  \@closing\\[3em]%
  \includegraphics[width=25mm]{signature}\\% Insert signature
  {\bfseries \@firstname~\@lastname}%
  \ifthenelse{\isundefined{\@enclosure}}{}{%
    \\%
    \vfill%
    {\color{color2}\itshape\enclname: \@enclosure}}}
\makeatother
%   to redefine the bibliography heading string ("Publications")
%\renewcommand{\refname}{Articles}

% bibliography with mutiple entries
%\usepackage{multibib}
%\newcites{book,misc}{{Books},{Others}}
%----------------------------------------------------------------------------------
%            content
%----------------------------------------------------------------------------------
\begin{document}
%\begin{CJK*}{UTF8}{gbsn}                          % to typeset your resume in Chinese using CJK
%-----       resume       ---------------------------------------------------------
\makecvtitle

%\section{Summary}

\section{Education}
\cventry{2022}{PhD Biology}{University of Oregon}{Eugene, OR}{}{}  % arguments 3 to 6 can be left empty
\cventry{2017}{MS Soil Science}{Penn State University}{State College, PA}{}{}
\cventry{2014}{BS Plant Sciences}{Cornell University}{Ithaca, NY}{}{}

%\section{Master thesis}
%\cvitem{title}{\emph{Title}}
%\cvitem{supervisors}{Supervisors}
%\cvitem{description}{Short thesis abstract}

\section{Experience}

%\subsection{Postdoctoral Work}
\cventry{2022--present}{Postdoctoral Scholar}{University of
  Oregon}{Eugene, OR}{}{
  \begin{itemize}
      \item Developed a novel framework for understanding microbiome heritabiliy in
          humans and animals, manuscript in prep. 
      \item Submitted a first-author publication to \textit{ISME Communications}.
      \item Presented original research on the heritability and function of the
          microbiome at the \textit{International Society of Microbial Ecology}
          meeting in Lausanne, Switzerland.
\end{itemize}
}
%\subsection{Graduate School}
    \cventry{2017--2022}{NSF Graduate Research Fellow}{University of
  Oregon}{Eugene, OR}{}{
  \begin{itemize}
      \item Assembled microbial genomes and performed comparative genomics
          across ecosystems using microbiome metagenomic data.
    \item Engineered microbiomes that consume greenhouse gases at high
        rates using laboratory selection experiments.
    \item Published a first-author paper on integrating quantitative genetics
        with microbiome science in \textit{Philosophical Transactions of the
          Royal Society}.
    \item Awarded multiple grants and fellowships including the
        National Science Foundation Graduate Research Fellowship and named an
          ARCS Scholar by the
          Oregon chapter of the ARCS Foundation. 
    \item Presented research to members and potential donors at the Oregon ARCS
        Foundation meeting.
\end{itemize}
}

\cventry{2015--2017}{Graduate Research Assistant}{Penn State University}{State
College, PA}{}{
\begin{itemize}
  \item Demonstrated a new approach to reducing the impact of agriculture on
      climate change by decreasing microbiome-mediated nitrous oxide emissions.
  \item Led field teams and mentored undergraduate research assistants.
  \item Collaborated with a team of over 14 scientists and industry partners.
  \item Presented research at scientific meetings including \textit{The
      Ecological Society of America} and the \textit{Soil Science Society of
        America}. 
  \item Communicated results to non-science audiences including farmers,
      technicians, and extension educators.
\end{itemize}
}

%\subsection{}
%\cventry{2015}{Research Assistant}{University of Delaware}{Newark,
%  DE}{}{
%\begin{itemize}
%  \item Designed and built experimental rice paddies to study the effects of
%    arsenic on rice, which is a major global health challenge. 
%  \item Developed an affordable strategy to reduce arsenic contamination in rice by
%    amending soils with freely available rice husk ash - research that has been
%    cited over 60 times.
%  \item Led an educational field day for middle school students of color who had never been on a farm. The students learned where
%    their food comes from and grew their own rice plants.
%\end{itemize}
%}

%\section{Languages}
%\cvitemwithcomment{Language 1}{Skill level}{Comment}
%\cvitemwithcomment{Language 2}{Skill level}{Comment}
%\cvitemwithcomment{Language 3}{Skill level}{Comment}

%\clearpage

\section{Skills}
\cventry{}{Formal Training}{}{}{}{
\begin{itemize}
  \item Advanced biostatistics coursework with both frequentist and Bayesian inference
    using \texttt{R} and \texttt{Stan}.  
  \item Training in
    bioinformatics at the Marine Biology Laboratory in Woods Hole, MA using
        \texttt{R}, \texttt{Python}, and \texttt{QIIME2} to analyze deep marker gene and metagenomic
    data.  
  \item Intensive workshop in machine learning for image analysis using
    deep neural networks with \texttt{Keras} and \texttt{TensorFlow} through the
    University of Oregon Data Science Initiative. 
\end{itemize}
}
\cventry{}{Microbiome Analysis}{}{}{}{
\begin{itemize}
    \item 16S rRNA gene amplicon analyses using \texttt{R}, \texttt{phyloseq}, and \texttt{QIIME2}.  
    \item Short-read shotgun metagenomic analyses including read recruitment, gene calling, taxonomic and functional annotation, contig assembly, genome binning, and comparative genomics.
    \item Statistical analyses in \texttt{R} including linear mixed effects models with crossed, nested, and repeated measures designs as well as non-parametric tests (Kruskal-Wallis, Wilcoxon, etc.). 
    \item Microbiome analyses including ecological dissimilarity metrics (Bray-Curtis, Jaccard, UniFrac, etc.), PCoA and NMDS ordinations, PERMANOVA, Mantel tests, and differential abundance tests.
\end{itemize}
}
\cvdoubleitem{Coding}{\texttt{R}, \texttt{Bash}, \texttt{Python}, \texttt{git}, Github}{Computing}{Linux, Unix, HPC, cloud computing}

%\section{Interests}
%\cvitem{hobby 1}{Description}
%\cvitem{hobby 2}{Description}
%\cvitem{hobby 3}{Description}

%\section{Extra 1}
%\cvlistitem{Item 1}
%\cvlistitem{Item 2}
%\cvlistitem{Item 3. This item is particularly long and therefore normally spans over several lines. Did you notice the indentation when the line wraps?}
%
%\section{Extra 2}
%\cvlistdoubleitem{Item 1}{Item 4}
%\cvlistdoubleitem{Item 2}{Item 5\cite{book1}}
%\cvlistdoubleitem{Item 3}{Item 6. Like item 3 in the single column list before, this item is particularly long to wrap over several lines.}
%\clearpage

%\section{Awards}
%\cventry{2017-2022}{University of Oregon}{Graduate School}{}{}{  % arguments 3 to 6 can be left empty
%\begin{itemize}
%\item NSF Graduate Research Fellowship Award
%\item Oregon ARCS Foundation Scholar
%\item Elma Hendricks Scholarship
%\item William R. Sistrom Memorial Scholarship
%\end{itemize}
%}
%
%\cventry{2015-2017}{Penn State University}{Graduate School}{}{}{
%\begin{itemize}
%\item Distinguished Master's Thesis Award
%\item Annie's Sustainable Agriculture Scholarship
%\item Scarlet Graduate Fellowship in Watershed Stewardship Award
%\item Katherine Mabis McKenna Fellowship Award
%\end{itemize}
%}
%
%\cventry{2010-2014}{Cornell University and Ithaca College}{Undergraduate}{}{}{
%\begin{itemize}
%\item Hatch/Multistate Grant
%\item Flora Brown Award
%\end{itemize}
%}

% Publications from a BibTeX file without multibib
%  for numerical labels: \renewcommand{\bibliographyitemlabel}{\@biblabel{\arabic{enumiv}}}% CONSIDER MERGING WITH PREAMBLE PART
%  to redefine the heading string ("Publications"): \renewcommand{\refname}{Articles}
%\nocite{*}
%\bibliographystyle{abbrv}
%\bibliography{publications}                        % 'publications' is the name of a BibTeX file

\section{Selected Publications}

\cvline{2023}{\textbf{Morris, AH} and  Bohannan, BJM. Artificial ecosystem
selection reveals relationships between microbiome composition and ecosystem
function. In review at \textit{ISME Communications} }

\cvline{2021}{\textbf{Morris, AH}, Isbell, SA, Saha, D and Kaye, JP. Mitigating
nitrogen pollution with under‐sown legume‐grass cover crop mixtures in winter
cereals. \textit{Journal of Environmental Quality} }

%\cvline{2021}{Isbell, SA, Bradley, BA, \textbf{Morris, AH}, Wallace, JM, and
%Kaye, JP. Nitrogen dynamics in grain cropping systems integrating multiple
%ecologically- based management strategies. \textit{Ecosphere} }

\cvline{2020}{Meyer, KM, \textbf{Morris, AH}, Webster, K, Klein, A, Kroegerv,
ME, Meredith, LK, et al. Belowground changes to community structure alter
methane-cycling dynamics in Amazonia. \textit{Environment International} }

\cvline{2020}{Meyer, KM, Hopple, AM, Klein, A, \textbf{Morris, AH}, Bridgham,
SD, and Bohannan, BJM. Community structure–ecosystem function relationships in
the Congo Basin methane cycle depend on the physiological scale of function.
\textit{Molecular Ecology} }

\cvline{2020}{\textbf{Morris, AH}, Meyer, KM, and Bohannan, BJM. Linking
microbial communities to ecosystem functions: what we can learn from
genotype–phenotype mapping in organisms. \textit{Philosophical Transactions of
the Royal Society B: Biological Sciences} }

%\cvline{2016}{Seyfferth, AL, \textbf{Morris, AH}, Gill, R, Kearns, KA, Mann,
%JN, Paukett, M, and Leskanic, C. Soil Incorporation of Silica-Rich Rice Husk
%Decreases Inorganic Arsenic in Rice Grain.  \textit{Journal of Agricultural and
%Food Chemistry} }

\section{Selected Presentations}
\cvline{2022}{\textbf{Morris, AH}, Bohannan, BJM. Artificial ecosystem selection reveals relationships
between microbiome composition and ecosystem function.
  \textit{ISME Meeting}. Lausanne, Switzerland.
}

\cvline{2019}{\textbf{Morris, AH}, Meyer, KM, Bohannan, BJM. Linking microbial communities
to ecosystem functions: what we can learn from genotype-phenotype mapping in
organisms.
  \textit{Achievement Rewards for College Scientists Annual Luncheon}. Portland, OR.
}

\cvline{2017}{\textbf{Morris, AH}, Isbell, S, Kaye, JP. Improving nitrogen
retention of agroecosystems using interseeded cover crops.  \textit{Ecological
Society of America Meeting}. Portland, OR.
}

\cvline{2016}{\textbf{Morris, AH}, Isbell, S, Kaye, JP. Managing Inter-Seeded
Cover Crops and Tillage to Decrease Nitrate Leaching and Nitrous Oxide
Emissions from Agricultural Soils. \textit{Soil Science Society of America
Meeting}. Phoenix, AZ.
}

%\cvline{2016}{\textbf{Morris, AH}. Greenhouse gases in the Reduced-Tillage Organic Systems Experiment (ROSE). \textit{ROSE Annual Advisory Boad Meeting}. Pine Grove Mills, PA.
%}

\section{Teaching}
\cvline{2018}{Faculty, Juneau Icefield Research Program: Geobotany and Ecology}
\cvline{2018}{Guest Lecture on Evolutionary Processes, University of Oregon: Ecology and Evolution}
\cvline{2018}{Teaching Assistant, University of Oregon: Ecology and Evolution, Genetics and Molecular Biology, Cells}
%\cvline{2017}{Guest Instructor, Penn State University: Impacts of Changing Hydrology on Ecosystem Services in Glacial Systems}
\cvline{2017}{Teaching Assistant, Penn State University: Soil Science}

%\section{References}
%\begin{cvcolumns}
%  \cvcolumn{Thesis advisors and supervisors}{\begin{itemize}\item Dr. Brendan
%  J.M. Bohannan\item Dr. Jason P. Kaye\item Dr. Angelia L. Seyfferth\end{itemize}}
%%  \cvcolumn{Category 2}{Amongst others:\begin{itemize}\item Person 1, and\item Person 2\end{itemize}(more upon request)}
%%  \cvcolumn[0.5]{All the rest \& some more}{\textit{That} person, and \textbf{those} also (all available upon request).}
%\end{cvcolumns}

%\section{Writing Samples}
%
%\cventry{2019}{Linking microbial communities to ecosystem functions: what we can
%learn from genotype-phenotype mapping in organisms}{Research abstract on bioRxiv}{}{}{\url{https://www.biorxiv.org/content/10.1101/740373v1}}
%\cventry{2018}{Funded grant application for the William R. Sistrom
%Memorial Scholarship}{University of Oregon, Department of Biology}{}{}{\url{https://ahmorris.org/assets/Morris_Sistrom_Statement.pdf}}
%\cventry{2019}{Using blogdown with github pages}{R tutorial
%blog post}{}{}{\url{https://ahmorris.org/blog/2019/09/03/blogdown-on-github-pages/}}
%\cventry{}{See also my publication
%page}{}{}{}{\url{https://ahmorris.org/pages/pubs}}
%\cventry{2018}{Example script for an intro to R laboratory}{University of
%Oregon, Department of
%Biology}{}{}{\url{https://ahmorris.org/assets/ecoli_growth_curve.R}}


% Publications from a BibTeX file using the multibib package
%\section{Publications}
%\nocitebook{book1,book2}
%\bibliographystylebook{plain}
%\bibliographybook{publications}                   % 'publications' is the name of a BibTeX file
%\nocitemisc{misc1,misc2,misc3}
%\bibliographystylemisc{plain}
%\bibliographymisc{publications}                   % 'publications' is the name of a BibTeX file

\clearpage
%-----       letter       ---------------------------------------------------------
% recipient data
\recipient{Talent Acquisition}{Seed Health, Inc.\\Venice, CA 90291}
\date{March 11, 2023}
\opening{Dear Talent Acquisition at Seed Health,}
\closing{Sincerely,}
%\enclosure[Attached]{curriculum vit\ae{}}          % use an optional argument to use a string other than "Enclosure", or redefine \enclname
\makelettertitle

I am excited to apply for the Microbiome Data Scientist position at Seed Health.
I recently completed my PhD in Biology at the University of Oregon, and I am
eager to apply my knowledge of the microbiome and bioinformatics to improving
human and planetary health.

Throughout my career, I have not only worked independently in designing and
completing multi-stage research projects, but also have experience leading
multidisciplinary research teams and working with both national and
international collaborators. In these teams, I helped to create realistic
deadlines and ensured timely completion of deliverables. I have demonstrated
success in scientific communication through funded grant proposals,
presentations at national and international conferences, and publications in
peer-reviewed scientific journals. 

The goal of my dissertation was to identify microbial indicators of greenhouse
gas emissions from ecosystems and to develop mitigation strategies to promote
microbial species that would facilitate climate change adaptation. 
In this work, I used statistical techniques and bioinformatic
pipelines that are commonly applied to the clinical analysis of microbiological
data in human health studies. These included the integration of multivariate,
multi-omics datasets including 16S and metagenomic sequencing data. Using
metagenomic short-read sequences, I assembled microbial genomes and performed
comparative genomic analyses to identify microbial indicators of high
greenhouse gas emitting ecosystems. In addition, I performed microbiome engineering
experiments to assemble ``teams'' of microorganisms that consumed greenhouse
gases at a high rate. The results of this experiment will help develop
microbial therapeutics to promote climate change adaptation.

Going forward, I hope to answer similar questions in the context of human
health through this position as a Microbiome Data Scientist. In particular, I look
forward to the opportunity to develop microbiome diagnostics with clinical
applications and microbiome therapeutics for the treatment of disease and the
promotion of human health. I believe my experience as a microbiome scientist
and bioinformatician will make a valuable contribution to the research
objectives at Seed Health.

Thank you for considering my application. I look forward to discussing this
position further.


\makeletterclosing

%\clearpage\end{CJK*}                              % if you are typesetting your resume in Chinese using CJK; the \clearpage is required for fancyhdr to work correctly with CJK, though it kills the page numbering by making \lastpage undefined
\end{document}


%% end of file `template.tex'.
