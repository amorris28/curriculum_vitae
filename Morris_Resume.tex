%% start of file `template.tex'.
%% Copyright 2006-2015 Xavier Danaux (xdanaux@gmail.com).
%
% This work may be distributed and/or modified under the
% conditions of the LaTeX Project Public License version 1.3c,
% available at http://www.latex-project.org/lppl/.


\documentclass[10pt,letterpaper,sans]{moderncv}        % possible options include font size ('10pt', '11pt' and '12pt'), paper size ('a4paper', 'letterpaper', 'a5paper', 'legalpaper', 'executivepaper' and 'landscape') and font family ('sans' and 'roman')

% moderncv themes
\moderncvstyle{classic}                             % style options are 'casual' (default), 'classic', 'banking', 'oldstyle' and 'fancy'
\moderncvcolor{blue}                               % color options 'black', 'blue' (default), 'burgundy', 'green', 'grey', 'orange', 'purple' and 'red'
%\renewcommand{\familydefault}{\sfdefault}         % to set the default font; use '\sfdefault' for the default sans serif font, '\rmdefault' for the default roman one, or any tex font name
%\nopagenumbers{}                                  % uncomment to suppress automatic page numbering for CVs longer than one page
%\usepackage{color,hyperref}
%\definecolor{darkblue}{rgb}{0.0,0.0,0.3}
%\hypersetup{colorlinks,breaklinks,
%            linkcolor=darkblue,urlcolor=darkblue,
%            anchorcolor=darkblue,citecolor=darkblue}
% character encoding
\usepackage[utf8]{inputenc}                       % if you are not using xelatex ou lualatex, replace by the encoding you are using
%\usepackage{CJKutf8}                              % if you need to use CJK to typeset your resume in Chinese, Japanese or Korean

% adjust the page margins
\usepackage[scale=0.8]{geometry}
%\setlength{\hintscolumnwidth}{3cm}                % if you want to change the width of the column with the dates
%\setlength{\makecvtitlenamewidth}{10cm}           % for the 'classic' style, if you want to force the width allocated to your name and avoid line breaks. be careful though, the length is normally calculated to avoid any overlap with your personal info; use this at your own typographical risks...

% personal data
\name{Andrew}{Morris}
\title{NSF Graduate Research Fellow}                               % optional, remove / comment the line if not wanted
%\address{1573 Wilson St.}{Eugene, OR 97402}{USA}% optional, remove / comment the line if not wanted; the "postcode city" and "country" arguments can be omitted or provided empty
\phone[mobile]{+1~(860)~670~4130}                   % optional, remove / comment the line if not wanted; the optional "type" of the phone can be "mobile" (default), "fixed" or "fax"
%\phone[fixed]{+2~(345)~678~901}
%\phone[fax]{+3~(456)~789~012}
\email{amorris3@uoregon.edu}                               % optional, remove / comment the line if not wanted
\homepage{ahmorris.org}                         % optional, remove / comment the line if not wanted
\social[linkedin]{andrew-morris-71033a41}                        % optional, remove / comment the line if not wanted
\social[twitter]{ahmorris1}                             % optional, remove / comment the line if not wanted
\social[github]{amorris28}                              % optional, remove / comment the line if not wanted
%\extrainfo{additional information}                 % optional, remove / comment the line if not wanted
\photo[64pt][0.4pt]{old_avatar}                       % optional, remove / comment the line if not wanted; '64pt' is the height the picture must be resized to, 0.4pt is the thickness of the frame around it (put it to 0pt for no frame) and 'picture' is the name of the picture file
%\quote{Some quote}                                 % optional, remove / comment the line if not wanted

% bibliography adjustements (only useful if you make citations in your resume, or print a list of publications using BibTeX)
%   to show numerical labels in the bibliography (default is to show no labels)
\makeatletter\renewcommand*{\bibliographyitemlabel}{\@biblabel{\arabic{enumiv}}}\makeatother
\makeatletter
\renewcommand*{\makeletterclosing}{
  \@closing\\[3em]%
  \includegraphics[width=25mm]{signature}\\% Insert signature
  {\bfseries \@firstname~\@lastname}%
  \ifthenelse{\isundefined{\@enclosure}}{}{%
    \\%
    \vfill%
    {\color{color2}\itshape\enclname: \@enclosure}}}
\makeatother
%   to redefine the bibliography heading string ("Publications")
%\renewcommand{\refname}{Articles}

% bibliography with mutiple entries
%\usepackage{multibib}
%\newcites{book,misc}{{Books},{Others}}
%----------------------------------------------------------------------------------
%            content
%----------------------------------------------------------------------------------
\begin{document}
%\begin{CJK*}{UTF8}{gbsn}                          % to typeset your resume in Chinese using CJK
%-----       resume       ---------------------------------------------------------
\makecvtitle

%\section{Summary}
%Biologist and data scientist asking and answering fundamental
%biological questions. I take a tool-agnostic approach leveraging machine
%learning and statistics to make discoveries
%from deep marker gene and metagenomic sequencing data. I
%am passionate about communication and reproducible research and demonstrate this
%with a track record of published peer-reviewed articles and funded grant
%proposals as well as maintaining code and data in public repositories. I have
%collaborated with research teams of three to fourteen members made up of diverse
%stakeholders and led multi-year research efforts. Finally, I prioritize Deep Work for focused productivity.

\section{Education}
\cventry{2022}{PhD Biology}{University of Oregon}{Eugene, OR}{}{}  % arguments 3 to 6 can be left empty
\cventry{2017}{MS Soil Science}{Penn State University}{State College, PA}{}{}
\cventry{2014}{BS Plant Sciences}{Cornell University}{Ithaca, NY}{}{}

%\section{Master thesis}
%\cvitem{title}{\emph{Title}}
%\cvitem{supervisors}{Supervisors}
%\cvitem{description}{Short thesis abstract}

\section{Experience}
%\subsection{Graduate School}
\cventry{2017--present}{NSF Graduate Research Fellow}{University of
Oregon}{Eugene, OR}{}{}
\cvlistitem{Identified members of the soil microbiome that decrease greenhouse
    gas emissions.}
      \cvlistitem{Analyzed marker gene and metagenomic data using random forest,
      multiple regression, and principal component analyses.}
      \cvlistitem{Wrote manuscripts and presentations using R Markdown and
      \LaTeX.}
    \cvlistitem{Awarded multiple grants and fellowships to fund my research.}
      \cvlistitem{Authored six peer-reviewed scientific papers.}


\cventry{2015--2017}{Graduate Research Assistant}{Penn State University}{State
College, PA}{}{}
  \cvlistitem{Demonstrated a strategy to reduce the impact of agriculture on climate
    change.}
    \cvlistitem{Analyzed a complex experimental design using mixed-effect models
    with nesting.}
    \cvlistitem{Collaborated with a team of over 14 people
    including scientists, technicians, educators, and farmers.}
    \cvlistitem{Communicated technical concepts to diverse audiences ranging from
  field-based teaching in glacial ecosystems in Alaska and Peru to farmer field
days in central Pennsylvania.}

%\subsection{}
\cventry{2015}{Research Assistant}{University of Delaware}{Newark,
DE}{}{}
\cvlistitem{Designed and built experimental rice paddies to study the effects of
arsenic on rice, which is a major global health challenge.}
\cvlistitem{Developed an affordable strategy to reduce arsenic contamination in
rice.}
\cvlistitem{Led an educational field day for middle school students of color who had never been on a farm. The students learned where
their food comes from and grew their own rice plants.}


%\section{Languages}
%\cvitemwithcomment{Language 1}{Skill level}{Comment}
%\cvitemwithcomment{Language 2}{Skill level}{Comment}
%\cvitemwithcomment{Language 3}{Skill level}{Comment}

%\section{Interests}
%\cvitem{hobby 1}{Description}
%\cvitem{hobby 2}{Description}
%\cvitem{hobby 3}{Description}

%\section{Extra 1}
%\cvlistitem{Item 1}
%\cvlistitem{Item 2}
%\cvlistitem{Item 3. This item is particularly long and therefore normally spans over several lines. Did you notice the indentation when the line wraps?}
%
%\section{Extra 2}
%\cvlistdoubleitem{Item 1}{Item 4}
%\cvlistdoubleitem{Item 2}{Item 5\cite{book1}}
%\cvlistdoubleitem{Item 3}{Item 6. Like item 3 in the single column list before, this item is particularly long to wrap over several lines.}
%\clearpage

\section{Awards}
\cvline{2017-2021}{\textbf{University of Oregon}}
  \cvlistdoubleitem{NSF Graduate Research Fellowship Award}{Oregon ARCS Foundation Scholar}
  \cvlistdoubleitem{Elma Hendricks Scholarship}{William R. Sistrom Memorial Scholarship}

\cvline{2015-2017}{\textbf{Penn State University}}
  \cvlistdoubleitem{Distinguished Master's Thesis Award}{Annie's Sustainable Agriculture Scholarship}
  \cvlistdoubleitem{Scarlet Graduate Fellowship in Watershed Stewardship Award}{Katherine Mabis McKenna Fellowship Award}

\cvline{2010-2014}{\textbf{Cornell University and Ithaca College}}
  \cvlistdoubleitem{Hatch/Multistate Grant}{Flora Brown Award}

  \clearpage

\section{Skills}
\cvdoubleitem{Typesetting}{\texttt{R Markdown}, \LaTeX, Jupyter}{Coding}{\texttt{R}, \texttt{Python}, \texttt{Bash}}
\cvdoubleitem{Computing}{Unix, Slurm, conda, modules}{Collaboration}{\texttt{git}, Github,
Slack, Zoom}
%\cvitem{Data Science}{QIIME2, TensorFlow, Keras, Stan}
% Publications from a BibTeX file without multibib
%  for numerical labels: \renewcommand{\bibliographyitemlabel}{\@biblabel{\arabic{enumiv}}}% CONSIDER MERGING WITH PREAMBLE PART
%  to redefine the heading string ("Publications"): \renewcommand{\refname}{Articles}
%\nocite{*}
%\bibliographystyle{abbrv}
%\bibliography{publications}                        % 'publications' is the name of a BibTeX file

\section{Publications}

\cvline{2021}{\textbf{Morris, AH}, Isbell, SA, Saha, D and Kaye, JP. Mitigating
  nitrogen pollution with under‐sown legume‐grass cover crop mixtures in winter
  cereals. \textit{Journal of Environmental Quality}.
\href{https://www.doi.org/10.1002/jeq2.20193}{https://doi.org/10.1002/jeq2.20193}}

\cvline{2021}{Isbell SA, Bradley BA, \textbf{Morris AH}, Wallace JM, Kaye JP.  Nitrogen
dynamics in grain cropping systems integrating multiple ecologically based management
strategies. \textit{Ecosphere}.
\href{https://doi.org/10.1002/ecs2.3380}{https://doi.org/10.1002/ecs2.3380}}

\cvline{2020}{Meyer KM, \textbf{Morris AH}, Webster K, Klein AM, Kroeger ME,
  Meredith LK, \textellipsis, Bohannan BJM.  Belowground changes to community structure alter methane-cycling
dynamics in Amazonia. \textit{Environment International}.
\href{https://doi.org/10.1016/j.envint.2020.106131}{https://doi.org/10.1016/j.envint.2020.106131}}

\cvline{2020}{Meyer KM, Hopple AM, Klein AM, \textbf{Morris AH}, Bridgham
SD, Bohannan BJM. Community structure – ecosystem function
relationships in the Congo Basin methane cycle depend on the physiological
scale of function. \textit{Molecular Ecology}.
\href{https://doi.org/10.1111/mec.15442}{https://doi.org/10.1111/mec.15442}}

\cvline{2020}{\textbf{Morris, AH}, Meyer, KM, and Bohannan, BJM. Linking microbial communities to ecosystem functions: what we can
  learn from genotype–phenotype mapping in organisms. \textit{Philosophical
    Transactions of the Royal
  Society B: Biological Sciences}.
\href{https://www.doi.org/10.1098/rstb.2019.0244}{https://doi.org/10.1098/rstb.2019.0244}}

%\cvline{2016}{Seyfferth, AL, \textbf{Morris, AH}, Gill, R,
%  Kearns, KA, Mann, JN, Paukett, M, and Leskanic, C. Soil Incorporation of
%  Silica-Rich Rice Husk Decreases Inorganic Arsenic in Rice Grain.
%  \textit{Journal of Agricultural and Food Chemistry}, 64(19), 3760-3766.
%\href{https://www.doi.org/10.1021/acs.jafc.6b01201}{https://doi.org/10.1021/acs.jafc.6b01201}}

\section{Presentations}
  \cvline{2019}{\textbf{Morris, A. H.}, Meyer, K. M., Bohannan, B. J. M.  Linking
  microbial communities to ecosystem functions: what we can learn from
genotype-phenotype mapping in organisms. Achievement Rewards for College
Scientists Annual Luncheon. Portland, OR.}
\cvline{2017}{\textbf{Morris, A. H.}, Isbell, S., Kaye, J.  Improving nitrogen retention of
  agroecosystems using interseeded cover crops. Ecological Society of America
  Meeting.
Portland, OR.}
\cvline{2017}{\textbf{Morris, A. H.}, Isbell, S., Kaye, J. Mitigating nitrogen
  pollution by interseeding cover crops into spelt. Sustainable Agriculture
Cropping Systems Symposium. State College, PA.}
\cvline{2016}{\textbf{Morris, A. H.}, Kaye, J. P. Managing Inter-Seeded Cover Crops
  and Tillage to Decrease Nitrate Leaching and Nitrous Oxide Emissions from
  Agricultural Soils. Soil Science Society of America Meeting. Phoenix,
Arizona.}
\cvline{2016}{\textbf{Morris, A. H.}, Isbell, S., Kaye, J. Kemanian, A. Managing cover
  crops and tillage to decrease nitrogen pollution from organically managed
  soils in Pennsylvania. Sustainable Agriculture Cropping Systems Symposium.
State College, PA.}
\cvline{2016}{\textbf{Morris, A. H.} Greenhouse gases in the Reduced-Tillage Organic
  Systems Experiment (ROSE). ROSE Annual Advisory Board Meeting. Pine Grove
Mills, PA.}

\section{Teaching}
\cvline{2018}{Faculty, Juneau Icefield Research Program: Geobotany and Ecology}
%  \cvline{2018}{Guest Lecture, University of Oregon: Ecology and Evolution,
%Evolutionary Processes}
  \cvline{2017--2018}{Teaching Assistant, University of Oregon: Cells; Genetics
  and Molecular Biology; Ecology and Evolution}
  \cvline{2017}{Guest Instructor, Penn State University: Impacts of Changing
  Hydrology on Ecosystem Services in Glacial Systems}
  \cvline{2017}{Teaching Assistant, Penn State University: Soil Science}

%\section{References}
%\begin{cvcolumns}
%  \cvcolumn{Thesis advisors and supervisors}{\begin{itemize}\item Dr. Brendan
%  J.M. Bohannan\item Dr. Jason P. Kaye\item Dr. Angelia L. Seyfferth\end{itemize}}
%%  \cvcolumn{Category 2}{Amongst others:\begin{itemize}\item Person 1, and\item Person 2\end{itemize}(more upon request)}
%%  \cvcolumn[0.5]{All the rest \& some more}{\textit{That} person, and \textbf{those} also (all available upon request).}
%\end{cvcolumns}

%\section{Writing Samples}
%
%\cventry{2019}{Linking microbial communities to ecosystem functions: what we can
%learn from genotype-phenotype mapping in organisms}{Research abstract on bioRxiv}{}{}{\url{https://www.biorxiv.org/content/10.1101/740373v1}}
%\cventry{2018}{Funded grant application for the William R. Sistrom
%Memorial Scholarship}{University of Oregon, Department of Biology}{}{}{\url{https://ahmorris.org/assets/Morris_Sistrom_Statement.pdf}}
%\cventry{2019}{Using blogdown with github pages}{R tutorial
%blog post}{}{}{\url{https://ahmorris.org/blog/2019/09/03/blogdown-on-github-pages/}}
%\cventry{}{See also my publication
%page}{}{}{}{\url{https://ahmorris.org/pages/pubs}}
%\cventry{2018}{Example script for an intro to R laboratory}{University of
%Oregon, Department of
%Biology}{}{}{\url{https://ahmorris.org/assets/ecoli_growth_curve.R}}


% Publications from a BibTeX file using the multibib package
%\section{Publications}
%\nocitebook{book1,book2}
%\bibliographystylebook{plain}
%\bibliographybook{publications}                   % 'publications' is the name of a BibTeX file
%\nocitemisc{misc1,misc2,misc3}
%\bibliographystylemisc{plain}
%\bibliographymisc{publications}                   % 'publications' is the name of a BibTeX file

%\clearpage
%%-----       letter       ---------------------------------------------------------
%% recipient data
%\recipient{Akosua Boateng and Alison Hill}{RStudio, Inc.\\250 Northern Ave\\Boston, MA 02210}
%\date{March 12, 2021}
%\opening{Dear Akosua Boateng and Alison Hill,}
%\closing{Sincerely,}
%%\enclosure[Attached]{curriculum vit\ae{}}          % use an optional argument to use a string other than "Enclosure", or redefine \enclname
%\makelettertitle
%
%I am writing regarding the R Markdown Internship at RStudio. I am a Biologist
%and Data Scientist at the University of Oregon currently pursuing my PhD. The
%reason I am applying for this internship is I am hoping to make a transition out
%of academia into the data science industry. This internship would be an
%excellent opportunity for me to get industry experience and to contribute to the
%development of tools that I know and love. I have been an R user for the last
%five years and I use R Markdown to write articles, give presentations, and
%publish my website. I have enjoyed learning and teaching the various R
%tools from RStudio, which gives me insights into the many powerful
%use cases and potential areas for growth that R Markdown has to offer.
%
%I am
%passionate about science communication with a track record of published
%articles and presentations to diverse audiences including K-12
%students, undergraduates, farmers, and scientists. In addition, I prioritize
%reproducible research using tools such as \texttt{git} and Github
%to maintain code and data in public repositories. You can see an example of this
%in the Github
%repository I have provided that includes a manuscript, a
%presentation, and an analysis that I have created using R Markdown documents.
%
%In the last six years as a graduate researcher, I have collaborated with local
%and remote teams made up of professors, technicians, educators,
%and students. As a result, I am comfortable working asynchronously with distributed teams collaborating over Github, Slack, and Zoom. In addition, as a PhD student
%I am experienced with long-term project management and self-motivation while
%working toward distant goals. Finally, I prioritize Deep Work for focused
%productivity by scheduling long periods of reading, writing, and coding.
%
%I would be thrilled to contribute  to R Markdown's development and to learn and
%grow as an R Studio team member. Thank you for taking the time to consider my
%application and I look forward to hearing from you.
%
%
%\makeletterclosing
%
%\clearpage\end{CJK*}                              % if you are typesetting your resume in Chinese using CJK; the \clearpage is required for fancyhdr to work correctly with CJK, though it kills the page numbering by making \lastpage undefined
\end{document}


%% end of file `template.tex'.
