\documentclass[10pt]{article}


% This is a helpful package that puts math inside length specifications
\usepackage{calc}
\usepackage{comment}
\usepackage{etaremune}

% Simpler bibsection for CV sections
% (thanks to natbib for inspiration)
\makeatletter
\newlength{\bibhang}
\setlength{\bibhang}{1em} %1em}
\newlength{\bibsep}
 {\@listi \global\bibsep\itemsep \global\advance\bibsep by\parsep}
\newenvironment{bibsection}%
        {\begin{etaremune}{}{%
%        {\begin{list}{}{%
       \setlength{\leftmargin}{\bibhang}%
       \setlength{\itemindent}{-\leftmargin}%
       \setlength{\itemsep}{\bibsep}%
       \setlength{\parsep}{\z@}%
        \setlength{\partopsep}{0pt}%
        \setlength{\topsep}{0pt}}}
        {\end{etaremune}\vspace{-.6\baselineskip}}
%        {\end{list}\vspace{-.6\baselineskip}}
\makeatother

% Layout: Puts the section titles on left side of page
\reversemarginpar

%
%         PAPER SIZE, PAGE NUMBER, AND DOCUMENT LAYOUT NOTES:
%
% The next \usepackage line changes the layout for CV style section
% headings as marginal notes. It also sets up the paper size as either
% letter or A4. By default, letter was used. If A4 paper is desired,
% comment out the letterpaper lines and uncomment the a4paper lines.
%
% As you can see, the margin widths and section title widths can be
% easily adjusted.
%
% ALSO: Notice that the includefoot option can be commented OUT in order
% to put the PAGE NUMBER *IN* the bottom margin. This will make the
% effective text area larger.
%
% IF YOU WISH TO REMOVE THE ``of LASTPAGE'' next to each page number,
% see the note about the +LP and -LP lines below. Comment out the +LP
% and uncomment the -LP.
%
% IF YOU WISH TO REMOVE PAGE NUMBERS, be sure that the includefoot line
% is uncommented and ALSO uncomment the \pagestyle{empty} a few lines
% below.
%

%% Use these lines for letter-sized paper
\usepackage[paper=letterpaper,
            %includefoot, % Uncomment to put page number above margin
            marginparwidth=1.2in,     % Length of section titles
            marginparsep=.05in,       % Space between titles and text
            margin=1in,               % 1 inch margins
            includemp]{geometry}

%% Use these lines for A4-sized paper
%\usepackage[paper=a4paper,
%            %includefoot, % Uncomment to put page number above margin
%            marginparwidth=30.5mm,    % Length of section titles
%            marginparsep=1.5mm,       % Space between titles and text
%            margin=25mm,              % 25mm margins
%            includemp]{geometry}

%% More layout: Get rid of indenting throughout entire document
\setlength{\parindent}{0in}

\usepackage[shortlabels]{enumitem}

%% Reference the last page in the page number
%
% NOTE: comment the +LP line and uncomment the -LP line to have page
%       numbers without the ``of ##'' last page reference)
%
% NOTE: uncomment the \pagestyle{empty} line to get rid of all page
%       numbers (make sure includefoot is commented out above)
%
\usepackage{fancyhdr,lastpage}
\pagestyle{fancy}
%\pagestyle{empty}      % Uncomment this to get rid of page numbers
\fancyhf{}\renewcommand{\headrulewidth}{0pt}
\fancyfootoffset{\marginparsep+\marginparwidth}
\newlength{\footpageshift}
\setlength{\footpageshift}
          {0.5\textwidth+0.5\marginparsep+0.5\marginparwidth-2in}
\lfoot{\hspace{\footpageshift}%
       \parbox{4in}{\, \hfill %
                    \arabic{page} of \protect\pageref*{LastPage} % +LP
%                    \arabic{page}                               % -LP
                    \hfill \,}}

% Finally, give us PDF bookmarks
\usepackage{color,hyperref}
\definecolor{darkblue}{rgb}{0.0,0.0,0.3}
\hypersetup{colorlinks,breaklinks,
            linkcolor=darkblue,urlcolor=darkblue,
            anchorcolor=darkblue,citecolor=darkblue}

%%%%%%%%%%%%%%%%%%%%%%%% End Document Setup %%%%%%%%%%%%%%%%%%%%%%%%%%%%


%%%%%%%%%%%%%%%%%%%%%%%%%%% Helper Commands %%%%%%%%%%%%%%%%%%%%%%%%%%%%

% The title (name) with a horizontal rule under it
% (optional argument typesets an object right-justified across from name
%  as well)
%
% Usage: \makeheading{name}
%        OR
%        \makeheading[right_object]{name}
%
% Place at top of document. It should be the first thing.
% If ``right_object'' is provided in the square-braced optional
% argument, it will be right justified on the same line as ``name'' at
% the top of the CV. For example:
%
%       \makeheading[\emph{Curriculum vitae}]{Your Name}
%
% will put an emphasized ``Curriculum vitae'' at the top of the document
% as a title. Likewise, a picture could be included:
%
%   \makeheading[\includegraphics[height=1.5in]{my_picutre}]{Your Name}
%
% the picture will be flush right across from the name.
\newcommand{\makeheading}[2][]%
        {\hspace*{-\marginparsep minus \marginparwidth}%
         \begin{minipage}[t]{\textwidth+\marginparwidth+\marginparsep}%
             {\large \bfseries #2 \hfill #1}\\[-0.15\baselineskip]%
                 \rule{\columnwidth}{1pt}%
         \end{minipage}}

% The section headings
%
% Usage: \section{section name}
\renewcommand{\section}[1]{\pagebreak[3]%
    \hyphenpenalty=10000%
    \vspace{1.3\baselineskip}%
    \phantomsection\addcontentsline{toc}{section}{#1}%
    \noindent\llap{\scshape\smash{\parbox[t]{\marginparwidth}{\raggedright #1}}}%
    \vspace{-\baselineskip}\par}

% An itemize-style list with lots of space between items
\newenvironment{outerlist}[1][\enskip\textbullet]%
        {\begin{itemize}[#1,leftmargin=*]}{\end{itemize}%
         \vspace{-.6\baselineskip}}

% An environment IDENTICAL to outerlist that has better pre-list spacing
% when used as the first thing in a \section
\newenvironment{lonelist}[1][\enskip\textbullet]%
        {\begin{list}{#1}{%
        \setlength{\partopsep}{0pt}%
        \setlength{\topsep}{0pt}}}
        {\end{list}\vspace{-.6\baselineskip}}

% An itemize-style list with little space between items
\newenvironment{innerlist}[1][\enskip\textbullet]%
        {\begin{itemize}[#1,leftmargin=*,parsep=0pt,itemsep=0pt,topsep=0pt,partopsep=0pt]}
        {\end{itemize}}

% An environment IDENTICAL to innerlist that has better pre-list spacing
% when used as the first thing in a \section
\newenvironment{loneinnerlist}[1][\enskip\textbullet]%
        {\begin{itemize}[#1,leftmargin=*,parsep=0pt,itemsep=0pt,topsep=0pt,partopsep=0pt]}
        {\end{itemize}\vspace{-.6\baselineskip}}

% To add some paragraph space between lines.
% This also tells LaTeX to preferably break a page on one of these gaps
% if there is a needed pagebreak nearby.
\newcommand{\blankline}{\quad\pagebreak[3]}
\newcommand{\halfblankline}{\quad\vspace{-0.5\baselineskip}\pagebreak[3]}

% Uses hyperref to link DOI
\newcommand\doilink[1]{\href{http://dx.doi.org/#1}{#1}}
\newcommand\doi[1]{doi:\doilink{#1}}

% For \url{SOME_URL}, links SOME_URL to the url SOME_URL
\providecommand*\url[1]{\href{#1}{#1}}
% Same as above, but pretty-prints SOME_URL in teletype fixed-width font
\renewcommand*\url[1]{\href{#1}{\texttt{#1}}}

% For \email{ADDRESS}, links ADDRESS to the url mailto:ADDRESS
\providecommand*\email[1]{\href{mailto:#1}{#1}}
% Same as above, but pretty-prints ADDRESS in teletype fixed-width font
%\renewcommand*\email[1]{\href{mailto:#1}{\texttt{#1}}}

%\providecommand\BibTeX{{\rm B\kern-.05em{\sc i\kern-.025em b}\kern-.08em
%    T\kern-.1667em\lower.7ex\hbox{E}\kern-.125emX}}
%\providecommand\BibTeX{{\rm B\kern-.05em{\sc i\kern-.025em b}\kern-.08em
%    \TeX}}
\providecommand\BibTeX{{B\kern-.05em{\sc i\kern-.025em b}\kern-.08em
    \TeX}}
\providecommand\Matlab{\textsc{Matlab}}

%%%%%%%%%%%%%%%%%%%%%%%% End Helper Commands %%%%%%%%%%%%%%%%%%%%%%%%%%%

%%%%%%%%%%%%%%%%%%%%%%%%% Begin CV Document %%%%%%%%%%%%%%%%%%%%%%%%%%%%

\begin{document}
%\makeheading{Andrew H. Morris}
\makeheading[\emph{Curriculum vitae}]{Andrew H. Morris}

\begin{comment}
\section{Contact Information}

% NOTE: Mind where the & separators and \\ breaks are in the following
%       table.
%
% ALSO: \rcollength is the width of the right column of the table
%       (adjust it to your liking; default is 1.85in).
%
\newlength{\rcollength}\setlength{\rcollength}{1.3in}%

\begin{tabular}[t]{@{}p{\textwidth-\rcollength}p{\rcollength}}
\href{http://www.cse.osu.edu/}%
     {Department of Computer Science and Engineering} & \\
\href{http://www.osu.edu/}{The Ohio State University}
%5289 University of Oregon   \hfill 860-670-4130
%
%335 Pacific Hall     \hfill \email{amorris3@uoregon.edu}
%
%Eugene, OR 97403
\end{tabular}
\end{comment}
%\section{Objective}

%Insert text here if you want to
%\begin{innerlist}
%\item More information and auxiliary documents can be found at\\\url{http://www.tedpavlic.com/facjobsearch/}
%\end{innerlist}

%\section{Research Interests}

%Microbial ecology, community ecology, biogeochemistry, soil science

\section{Education}

Ph.D. \textbf{University of Oregon}, Biology \hfill 2022

M.S. \textbf{Penn State University}, Soil Science \hfill 2017

B.S. \textbf{Cornell University}, Plant Sciences \hfill 2014

\section{Research Appointments}

\textbf{Postdoctoral Scholar}, University of Oslo \hfill {2024-Present}

\textbf{Postdoctoral Scholar}, University of Oregon \hfill {2022-2023}

\textbf{NSF Graduate Research Fellow} \hfill {2016-2021}

\textbf{ARCS Scholar} \hfill {2017-2020}

\textbf{Graduate Employee}, University of Oregon \hfill {2017-2018}

\textbf{Graduate Research Assistant}, Penn State University \hfill {2015-2017}

\textbf{Research Assistant}, University of Delaware \hfill {2015}

\section{Publications}
\vspace{-.1275in}
\begin{bibsection}

    \item {\bf Morris, A. H.} and Bohannan, B. J. M. 2024.
        ``Quantitative estimates of microbiome heritability and their implications.'' 
        \emph{Nature Microbiology} \doi{10.1038/s41564-024-01865-w}

    \item {\bf Morris, A. H.}, Isbell, S. A., Saha, D., and Kaye, J. P. 2021.
	    ``Mitigating nitrogen pollution with undersown legume-grass cover
	    crop mixtures in winter cereals'' \emph{Journal of
	Environmental Quality} \doi{10.1002/jeq2.20193}
	%\href{https://doi.org/10.1002/jeq2.20193}{doi:10.1002/jeq2.20193}

    \item Isbell, S. A., Bradley, B. A., {\bf Morris, A. H.}, Wallace, J. M.,
      Kaye, J. P. 2021. ``Nitrogen dynamics in grain cropping systems integrating
	multiple ecologically-based management strategies'' \emph{Ecosphere} \doi{10.1002/ecs2.3380}
%      \href{ https://doi.org/10.1002/ecs2.3380}{doi:10.1002/ecs2.3380}
      
    \item 
      Meyer, K. M., {\bf Morris, A. H.}, Webster, K., Klein, A., Kroegerv, M.
      E., Meredith, L. K., Brændholt, A., Nakamurat, F., Venturini, A., Fonseca
      de Souzat, L., Shek, K. L., Danielson, R., van Haren, J., Barbosa de
      Camargot, P., Tsait, S. M., Dini-Andreote, F., Nüsslein, K., Saleska, S.
      R., Rodrigues, J. L. M., Bohannan, B. J. M. 2020. ``Belowground changes to
      community structure alter methane-cycling dynamics in Amazonia''
	\emph{Environment International} \\\doi{10.1016/j.envint.2020.106131}
%      \\\href{https://doi.org/10.1016/j.envint.2020.106131}{doi:10.1016/j.envint.2020.106131}

    \item Meyer, K. M., Hopple, A. M., Klein, A., {\bf Morris, A.H.}, Bridgham,
      S. D., Bohannan, B. J. M. 2020. ``Community structure\textendash ecosystem
      function relation-ships in the Congo Basin methane cycle depend on the physiological
	scale of function.'' \emph{Molecular Ecology}. \doi{10.1111/mec.15442} %\href{https://doi.org/10.1111/mec.15442}{doi:10.1111/mec.15442}

    \item {\bf Morris, A. H.}, Meyer, K. M., Bohannan, B. J. M. 2020. ``Linking microbial communities
      to ecosystem functions: what we can learn from genotype-phenotype mapping
      in organisms'' \emph{Philosophical Transactions of the
	Royal Society B}. \\\doi{10.1098/rstb.2019.0244}%\\\href{https://doi.org/10.1098/rstb.2019.0244}{doi:10.1098/rstb.2019.0244}

    \item Seyfferth, A. L., {\bf Morris, A. H.}, Gill, R., Kearns, K. A., Mann,
      J. N., Paukett, M., and Leskanic, C. 2016. ``Soil-incorporation of
      silica-rich rice husk decreases inorganic As in rice grain.''
      \emph{Journal of Agricultural and Food Chemistry}, \\64(19):3760--3766
	\doi{10.1021/acs.jafc.6b01201} %\href{https://doi.org/10.1021/acs.jafc.6b01201}{doi:10.1021/acs.jafc.6b01201}
\end{bibsection}

\section{Pre-prints}
\vspace{-.125in}
\begin{bibsection}
    \item Week, B., {\bf Morris, A. H.}, and Bohannan, B. J. M. 2024
        ``The Evolution of Microbiome-Mediated Traits.'' \emph{BioRxiv}. \\\doi{10.1101/2024.03.29.587374v1}

    \item {\bf Morris, A. H.} and Bohannan, B. J. M. 2023.
        ``Response of soil microbiome composition to selection on methane oxidation rate.'' \emph{BioRxiv} \\\doi{10.1101/2023.06.23.546315}
\end{bibsection}

\section{In Press}
\vspace{-.125in}
\begin{bibsection}
        \item Akdeniz B. C., {\bf Morris, A. H.}, Møller P., Andreassen O., Hovig E., and \\Dominguez-Valentin M. 
        ``Risk stratification of colorectal cancer using a combined effect of polygenic risk score and mismatch repair genes for Norwegian cohort.'' \emph{Tumori Journal}.
    \end{bibsection}
\begin{comment}

\section{In Review}
\vspace{-.125in}
\begin{bibsection}
\end{bibsection}
\end{comment}

% Add a little space to nudge next ``Conference Publications'' marginpar
% down to make room for tall ``Submitted Journal Publications''
% marginpar. If there are enough submitted journal publications, this
% space will not be needed (and should be removed).
% \vspace{0.1in}
%\begin{comment}
\section{In Prep}
\vspace{-.1in}
\begin{bibsection}
    \item {\bf Morris, A. H.}, Kyle M. Meyer, Bohannan, B. J. M., et al.
        ``Identifying the metagenomic drivers of methane emissions from pastures of the brazilian amazon.''
\end{bibsection}
%\end{comment}

\section{Awards and Grants}
\begin{comment}
Travel Awards
\begin{innerlist}
\item Workshop on Environmetrics, Raleigh, NC\hfill Oct 2012
\item Case Studies in Bayesian Statistics and\hfill Oct 2011\\
Machine Learning, Pittsburgh, PA
\item IMS/ISBA Joint International Meeting, Park City, UT\hfill Jan 2011
\end{innerlist}

\halfblankline
\end{comment}
University of Oregon, Post-doc
\begin{innerlist}
\item Contributed to funded NSF proposal \emph{Using Rules of Life to Capture Atmospheric Carbon: Interdisciplinary Convergence to Accelerate Research on Biological \\Sequestration (CARBS)}\hfill (\$3,000,000 USD) 2023
\end{innerlist}

\halfblankline

University of Oregon, Graduate School
\begin{innerlist}
\item Elma Hendricks Scholarship\hfill 2018
\item William R. Sistrom Memorial Scholarship\hfill 2018
\item Oregon Achievement Rewards for College Scientists (ARCS) Scholar\hfill 2017
\end{innerlist}

\halfblankline

The Pennsylvania State University, Graduate School
\begin{innerlist}
\item Distinguished Master's Thesis Award\hfill 2017
\item NSF Graduate Research Fellowship Award\hfill 2016
\item Annie's Sustainable Agriculture Scholarship\hfill 2016
\item Scarlet Graduate Fellowship in Watershed Stewardship Award\hfill 2015
\item Katherine Mabis McKenna Fellowship Award\hfill 2015
\end{innerlist}

\halfblankline

Cornell University and Ithaca College, Undergraduate
\begin{innerlist}
\item Hatch/Multistate Grant\hfill 2013
\item Flora Brown Award\hfill 2010
\end{innerlist}

\section{Presentations and Posters}
\begin{innerlist}
\item {\bf Morris, A. H.} and Bohannan, B. J. M. Quantitative estimates of microbiome heritability and their implications. Nordic Conference on Future Health. Trondheim, Norway. \hfill 2024
\item Bob Week, {\bf Morris, A. H.}, and Bohannan, B. J. M. The evolution of microbiome mediated traits. Joint Congress on Evolutionary Biology. Montreal, QC, Canada. \hfill 2024
\item {\bf Morris, A. H.} and Bohannan, B. J. M. Microbiome heritability and the evolution of host-level traits. Symbiosis Theory Workshop. Eugene, OR. \hfill 2023
\item {\bf Morris, A. H.} and Bohannan, B. J. M. Artificial ecosystem selection reveals
relationships between microbiome
composition and ecosystem function. ISME \\Meeting. Lausanne, Switzerland. \hfill 2022
\item {\bf Morris, A. H.}, Meyer, K. M., Bohannan, B. J. M.  Linking microbial
communities to ecosystem functions: what we can learn from genotype-phenotype
mapping in organisms. Achievement Rewards for College
  Scientists Annual Luncheon. Portland, OR. \hfill 2019
\item {\bf Morris, A. H.}, Isbell, S., Kaye, J.  Improving nitrogen retention of
  agroecosystems using interseeded cover crops. Ecological Society of America. Portland, OR. \hfill 2017
\item {\bf Morris, A. H.}, Isbell, S., Kaye, J. Mitigating nitrogen pollution by interseeding cover crops into spelt. Sustainable Agriculture Cropping Systems Symposium. State College, PA. \hfill 2017
\item {\bf Morris, A. H.}, Kaye, J. P. Managing Inter-Seeded Cover Crops and Tillage to Decrease Nitrate Leaching and Nitrous Oxide Emissions from Agricultural Soils. Soil Science Society of America Meeting. Phoenix, Arizona. \hfill 2016
\item {\bf Morris, A. H.}, Isbell, S., Kaye, J. Kemanian, A. Managing cover crops and tillage to decrease nitrogen pollution from organically managed soils in Pennsylvania. Sustainable Agriculture Cropping Systems Symposium. State College, PA. \hfill 2016
\item Isbell, S. and {\bf Morris, A. H.}. Nitrogen dynamics in cover crop-based reduced tillage cropping systems. Rodale Institute U.S.-Argentina Travel Program. Russell E. Larson Agricultural Research Center, Rock Springs, PA. \hfill May 2016
\item Saha, D. and {\bf Morris, A. H.}. Unraveling the interactive controls of tillage, residue, and manure additions on nitrous oxide emissions in grain and silage systems. Rodale Institute U.S.-Argentina Travel Program. Russell E. Larson Agricultural Research Center, Rock Springs, PA. \hfill May 2016
\item {\bf Morris, A. H.} Greenhouse gases in the Reduced-Tillage Organic
  Systems Exper-iment (ROSE). ROSE Annual Advisory Board Meeting. Pine Grove Mills, PA. Jan. \hfill 2016
\item Seyfferth, A. L., {\bf Morris, A. H.}, Kearns, K., Mann, J., Teasley, W., Limmer, M., Amaral, D.. Impacts of Increased Soil Si on Fe Mineral Composition and As Cycling in Rice Paddies. Soil Science Society of America Meeting. Minneapolis, Minnesota. \hfill 2015
\item Teasley, W, Seyfferth, A. L., {\bf Morris, A. H.}, Johansson, A. The Effect of Si Amendments on As Accumulation and Greenhouse Gas Emissions in Rice (Oryza sativa L). Soil Science Society of America Meeting. Minneapolis, Minnesota. \hfill 2015
\end{innerlist}



\section{Teaching Appointments}

%\textbf{Teaching Assistant} \hfill {Springs 2011--12}
\begin{innerlist}
\item[] Faculty, Juneau Icefield Research Program: Geobotany and Ecology
	\hfill 2018
\item[] Guest Lecture, University of Oregon: Ecology and Evolution, \\
  Evolutionary Processes \hfill 2018
\item[] Teaching Assistant, University of Oregon: Ecology and Evolution
	\hfill 2018
\item[] Teaching Assistant, University of Oregon: Genetics and
	Molecular Biology \hfill 2018
\item[] Teaching Assistant, University of Oregon: Cells \hfill 2017
\item[] Guest Instructor, Penn State University: Impacts of Changing
	Hydrology on\\Ecosystem Services in Glacial Systems \hfill 2017
\item[] Teaching Assistant, Penn State University: Soil Science \hfill
	2017

\end{innerlist}

\begin{comment}
Graduate Teaching Assistant \hfill {Spring 2018}
\begin{innerlist}

\item[] BI 283H - Honors Biology III: Ecology and Evolution\\
        Instructors: Brendan J. M. Bohannan, Ph.D and Laurel Pfeifer-Meister, Ph.D\\
        Department of Biology,\\
        University of Oregon

\end{innerlist}

Graduate Teaching Assistant \hfill {Winter 2018}
\begin{innerlist}

\item[] BI 282H - Honors Biology II: Genetics and Molecular Biology\\
        Instructors: Tory Herman, Ph.D and Laurel Pfeifer-Meister, Ph.D\\
        Department of Biology,\\
        University of Oregon

\end{innerlist}

Graduate Teaching Assistant \hfill {Fall 2017}
\begin{innerlist}
\item[] BI 211 - General Biology I: Cells\\
        Instructor: Cristin Hulslander, Ph.D\\
        Department of Biology,\\
        University of Oregon

\end{innerlist}


Visiting Instructor \hfill {Summer 2017}
\begin{innerlist}
\item[] CAUSE 2017 - Impacts of Changing Hydrology on Ecosystem Services in Glacial Systems\\
        Instructors: Denice Wardrop, Ph.D and Mike Nassry, Ph.D\\
        College of Earth and Mineral Sciences,\\
        The Pennsylvania State University

\end{innerlist}

Graduate Teaching Assistant \hfill {Spring 2017}
\begin{innerlist}
\item[] SOILS 102 - Introductory Soil Science Laboratory\\
        Instructor: Richard Stehouwer, Ph.D\\
        Department of Ecosystem Science and Management,\\
        The Pennsylvania State University

\end{innerlist}
\end{comment}

\section{Mentorship}
\begin{innerlist}
\item[] Graduate student peer mentor, Institute of Ecology and Evolution, University of Oregon \hfill 2020-2021
\item[] Rotation student mentor, Bohannan Lab, University of Oregon \hfill 2019
\item[] Undergraduate student mentor, Kaye Lab, Penn State University \hfill 2016
\item[] Undergraduate student mentor, Seyfferth Lab, University of Delaware \hfill 2015

\end{innerlist}
\section{Service}
\begin{innerlist}
\item[] Organizer of the Oslo Bioinformatics Workshop Week 2024, Oslo, Norway \hfill 2024
\item[] Student Volunteer at the Ecological Society of America meeting, Portland, OR \hfill 2017
\item[] Reviewer for American Naturalist, Environmental Microbiology, FEMS Microbiology Ecology, Molecular Ecology, Nature Ecology and Evolution, Scientific Data, Scientific Reports
\end{innerlist}




\begin{comment}
Recruiting Committee, Division of Biostatistics \hfill {May 2010 -- Present}
\begin{innerlist}
    \item Assist with planning of annual Division of Biostatistics Open House and Admitted Student Visit Days
    \item Meet with prospective and admitted students %; answer questions from a student's perspective
\end{innerlist}

\halfblankline

Student Member of Search Committee for the \hfill {June 2010 -- Aug 2010}\\
SPH Coordinator of Recruitment and Student Leadership
\begin{innerlist}
    \item Assisted in job search for the SPH Coordinator of Recruitment and Student Leadership
    \item Reviewed applications, conducted interviews
\end{innerlist}
\end{comment}

%\section{References}
\begin{comment}
Brendan J. M. Bohannan
\begin{innerlist}
\item[] Professor of Environmental Studies and Biology \hfill {Phone: 541-346-4883}\\
Department of Biology \hfill{E-mail: bohannan@uoregon.edu}\\
University of Oregon
\end{innerlist}

\halfblankline

Jason P. Kaye
\begin{innerlist}
\item[] Professor of Soil Biogeochemistry \hfill {Phone: 814-863-1614}\\
Department of Ecosystem Science and Management \hfill{E-mail: jpk12@psu.edu}\\
The Pennsylvania State University
\end{innerlist}
\end{comment}

%\section{Hardware and Software Skills}
\begin{comment}
Computer Programming:

\begin{innerlist}
    \item R, Unix/Linux
\end{innerlist}

\halfblankline
\end{comment}
\end{document}

%%%%%%%%%%%%%%%%%%%%%%%%%% End CV Document %%%%%%%%%%%%%%%%%%%%%%%%%%%%%

%----------------------------------------------------------------------%
% The following is copyright and licensing information for
% redistribution of this LaTeX source code; it also includes a liability
% statement. If this source code is not being redistributed to others,
% it may be omitted. It has no effect on the function of the above code.
%----------------------------------------------------------------------%
% Copyright (c) 2007, 2008, 2009, 2010, 2011 by Theodore P. Pavlic
%
% Unless otherwise expressly stated, this work is licensed under the
% Creative Commons Attribution-Noncommercial 3.0 United States License. To
% view a copy of this license, visit
% http://creativecommons.org/licenses/by-nc/3.0/us/ or send a letter to
% Creative Commons, 171 Second Street, Suite 300, San Francisco,
% California, 94105, USA.
%
% THE SOFTWARE IS PROVIDED "AS IS", WITHOUT WARRANTY OF ANY KIND, EXPRESS
% OR IMPLIED, INCLUDING BUT NOT LIMITED TO THE WARRANTIES OF
% MERCHANTABILITY, FITNESS FOR A PARTICULAR PURPOSE AND NONINFRINGEMENT.
% IN NO EVENT SHALL THE AUTHORS OR COPYRIGHT HOLDERS BE LIABLE FOR ANY
% CLAIM, DAMAGES OR OTHER LIABILITY, WHETHER IN AN ACTION OF CONTRACT,
% TORT OR OTHERWISE, ARISING FROM, OUT OF OR IN CONNECTION WITH THE
% SOFTWARE OR THE USE OR OTHER DEALINGS IN THE SOFTWARE.
%----------------------------------------------------------------------%
